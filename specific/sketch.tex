%\documentclass[tikz]{standalone}
%\usepackage{pgfmath}
%\usepackage{csvsimple}
%\usetikzlibrary{calc,arrows.meta}
%
%\begin{document}
\newlength{\intervalwidth}
\newlength{\controldistance}
\newlength{\yscale}
\setlength\intervalwidth{1cm}
\setlength\controldistance{.3cm}
\setlength\yscale{.4cm}
\def\tikzscale{0.80}
\pgfmathsetmacro\invtikzscale{1/\tikzscale}

% A simple empty decoration, that is used to ignore the last bit of the path
\pgfdeclaredecoration{ignore}{final}
{
\state{final}{}
}

% Declare the actual decoration.
\pgfdeclaremetadecoration{begin}{initial}{
    \state{initial}[
        width={\the\pgfdecorationsegmentlength},
        next state=final
    ]
    {\decoration{curveto}}

    \state{final}
    {\decoration{ignore}}
}

% Create a key for easy access to the decoration (as suggested by Jake).
\tikzset{begin segment/.style={decoration={begin},decorate, segment length=#1}}

{\normalsize
\begin{center}
\resizebox{0.8\textwidth}{!}{%
\begin{tikzpicture}
  \foreach\k in {0} {
  \pgfmathtruncatemacro\knext{\k+1}
  \useasboundingbox (-1em,1.25\yscale) rectangle (4.0\intervalwidth,8.5\yscale);
  \begin{scope}
  \clip (-1em,1.25\yscale) rectangle (3.5\intervalwidth,8.5\yscale);
  %\draw[xstep=\intervalwidth,ystep=\yscale] (0,0) grid (4\intervalwidth,8\yscale);

  \ifthenelse{\k = 0}{%
    \tikzset{rho/.style={shape=circle,inner sep=1pt,draw=KULeuvenBlauw,fill=KULeuvenBlauw}}
    \tikzset{rhobar/.style={shape=circle,inner sep=1pt,draw=KULeuvenBlauw,fill=white}}
    \tikzset{phi/.style={shape=circle,inner sep=1pt,draw=KULeuvenSecundair}}
    \tikzset{mu/.style={shape=circle,inner sep=1pt,draw=KULeuvenFaculteit,fill=white}}
    \tikzset{rholine/.style={KULeuvenBlauw,dashed,line width=1pt}}
    \tikzset{philine/.style={KULeuvenSecundair,line width=1pt}}
    \tikzset{muline/.style={KULeuvenFaculteit,line width=1pt}}
    \tikzset{jumpline/.style={black,|->,line width=1pt}}
  }{%
    \tikzset{rho/.style={shape=coordinate,inner sep=0pt,draw=KULeuvenBlauw,fill=KULeuvenBlauw}}
    \tikzset{rhobar/.style={shape=coordinate,inner sep=0pt,draw=KULeuvenBlauw,fill=white}}
    \tikzset{phi/.style={shape=coordinate,inner sep=0pt,draw=KULeuvenSecundair}}
    \tikzset{mu/.style={shape=coordinate,inner sep=0pt,draw=KULeuvenFaculteit,fill=white}}
    \tikzset{rholine/.style={begin segment=0.85\intervalwidth,KULeuvenBlauw,dashed,line width=0pt}}
    \tikzset{philine/.style={begin segment=0.75\intervalwidth,KULeuvenSecundair,line width=1pt}}
    \tikzset{muline/.style={begin segment=0.75\intervalwidth,KULeuvenFaculteit,line width=1pt}}
    \tikzset{jumpline/.style={black,|->,line width=1pt}}
  }

  \csvreader[head to column names]{specific/input.csv}
  {rho\k=\nrho,rho\k out=\nrhoout,phi\k=\nphi,phi\k in=\nphiin,%
  mu\k =\nmu,mu\k out=\nmuout,mu\k in=\nmuin,rho\knext=\nprho,rho\knext out=\nprhoout,rho\knext bar=\nprhobar,rho\knext barin=\nprhobarin}%
  {%
\pgfmathtruncatemacro\nnext{\thecsvrow}
\pgfmathtruncatemacro\n{\nnext-1}
\pgfmathtruncatemacro\nprev{\n-1}
\coordinate (rho\n) at (\n\intervalwidth, \nrho\yscale);
\coordinate (out rho\n) at (\n\intervalwidth + \controldistance, \nrho\yscale + \nrhoout\yscale);

\coordinate (phi\n) at (\n\intervalwidth, \nphi\yscale);
\coordinate (in  phi\n) at (\n\intervalwidth - \controldistance, \nphi\yscale - \nphiin\yscale);

\coordinate (mu\n) at (\n\intervalwidth, \nmu\yscale);
\coordinate (out mu\n) at (\n\intervalwidth + \controldistance, \nrho\yscale + \nmuout\yscale);
\coordinate (in  mu\n) at (\n\intervalwidth - \controldistance, \nmu\yscale  - \nmuin\yscale);

\coordinate (prho\n) at (\n\intervalwidth, \nprho\yscale);
\coordinate (out prho\n) at (\n\intervalwidth + \controldistance, \nprho\yscale + \nprhoout\yscale);
\coordinate (prhobar\n) at (\n\intervalwidth, \nprhobar\yscale);
\coordinate (in  prhobar\n) at (\n\intervalwidth - \controldistance, \nprhobar\yscale - \nprhobarin\yscale);

\ifthenelse{\n = 0}{}{%
  \draw [muline]  (rho\nprev) .. controls (out mu\nprev)  and (in mu\n)  .. (mu\n);
  \ifthenelse{\n > \k}{
    \draw [philine] (rho\nprev) .. controls (out rho\nprev) and (in phi\n) .. (phi\n);
    \draw [rholine]  (prho\nprev) .. controls (out prho\nprev)  and (in prhobar\n)  .. (prhobar\n);
  }{}
}
\node [rho] at (rho\n) {};
\node [phi] at (phi\n) {};
\node [mu]  at (mu\n)  {};
\node [rhobar] at (prhobar\n) {};
\node [rho] at (prho\n) {};
\ifthenelse{\k = 0}{
  \ifthenelse{\n > \k}{
    \draw [jumpline] (phi\n) -- (mu\n);
  rlet{orange}{KULeuvenLichtblauw}
    \draw [jumpline] (prhobar\n) -- (prho\n);
  }{}
}{} % end of conditional on k=0
  } % end of CSV reader
  \end{scope}
  \ifthenelse{\k = 0}{
    \tikzset{prop/.style={}}
    \node [color=KULeuvenFaculteit, above=.4\yscale of mu3]   {\footnotesize micro $\mathcal{X}$};
    \node [color=KULeuvenSecundair, below=.4\yscale of phi3]  {\footnotesize macro $\mathcal{Z}$};
    \node [color=KULeuvenBlauw,     left=.4\yscale of prhobar2] {\footnotesize macro $\rho$};
    \draw [->] (0,1.5\yscale) to node [above=.2em] {time} (3\intervalwidth,1.5\yscale);
    \draw [->] (0,1.5\yscale) to node [left=.2em,above,rotate=90] {value} (0,7\yscale);
    \node [anchor=south west] at (2.5\intervalwidth,2\yscale) {$k = \k$};
  }
} % end of foreach k
\end{tikzpicture}
%\end{document}
}
\end{center}
}
